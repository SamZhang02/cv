%------------------------s
% Resume in Latex
% Author : Jake Gutierrez
% Based off of: https://github.com/sb2nov/resume
% License : MIT
%------------------------

\documentclass[letterpaper,11pt]{article}
\usepackage{CJKutf8}

\usepackage{latexsym}
\usepackage[empty]{fullpage}
\usepackage{titlesec}
\usepackage{marvosym}
\usepackage[usenames,dvipsnames]{color}
\usepackage{verbatim}
\usepackage{enumitem}
\usepackage[hidelinks]{hyperref}
\usepackage{fancyhdr}
\usepackage[english]{babel}
\usepackage{tabularx}
\usepackage{setspace}
\input{glyphtounicode}

%----------FONT OPTIONS----------
% sans-serif
% \usepackage[sfdefault]{FiraSans}
% \usepackage[sfdefault]{roboto}
% \usepackage[sfdefault]{noto-sans}
% \usepackage[default]{sourcesanspro}

% serif
% \usepackage{CormorantGaramond}
% \usepackage{charter}

\onehalfspacing

\pagestyle{fancy}
\fancyhf{} % clear all header and footer fields
\fancyfoot{}
\renewcommand{\headrulewidth}{0pt}
\renewcommand{\footrulewidth}{0pt}

% Adjust margins
\addtolength{\oddsidemargin}{-0.5in}
\addtolength{\evensidemargin}{-0.5in}
\addtolength{\textwidth}{1in}
\addtolength{\topmargin}{-.5in}
\addtolength{\textheight}{1.0in}

\urlstyle{same}

\raggedbottom
\raggedright
\setlength{\tabcolsep}{0in}

% Sections formatting
\titleformat{\section}{
  \vspace{-4pt}\scshape\raggedright\large
}{}{0em}{}[\color{black}\titlerule \vspace{-5pt}]

% Ensure that generate pdf is machine readable/ATS parsable
\pdfgentounicode=1

%-------------------------
% Custom commands
\newcommand{\resumeItem}[1]{
  \item\small{
    {#1 \vspace{-2pt}}
  }
}

\newcommand{\resumeSubheading}[4]{
  \vspace{-2pt}\item
  \begin{tabular*}{0.97\textwidth}[t]{l@{\extracolsep{\fill}}r}
    \textbf{#1} & #2 \\
    \textit{\small#3} & \textit{\small #4} \\
  \end{tabular*}\vspace{-7pt}
}

\newcommand{\resumeSubSubheading}[2]{
  \item
  \begin{tabular*}{0.97\textwidth}{l@{\extracolsep{\fill}}r}
    \textit{\small#1} & \textit{\small #2} \\
  \end{tabular*}\vspace{-7pt}
}

\newcommand{\resumeProjectHeading}[2]{
  \item
  \begin{tabular*}{0.97\textwidth}{l@{\extracolsep{\fill}}r}
    \small#1 & #2 \\
  \end{tabular*}\vspace{-7pt}
}

\newcommand{\resumeSubItem}[1]{\resumeItem{#1}\vspace{-4pt}}

\renewcommand\labelitemii{$\vcenter{\hbox{\tiny$\bullet$}}$}

\newcommand{\resumeSubHeadingListStart}{\begin{itemize}[leftmargin=0.15in, label={}]}
    \newcommand{\resumeSubHeadingListEnd}{\end{itemize}}
\newcommand{\resumeItemListStart}{\begin{itemize}}
    \newcommand{\resumeItemListEnd}{\end{itemize}\vspace{-5pt}}

%-------------------------------------------
%%%%%%  RESUME STARTS HERE  %%%%%%%%%%%%%%%%%%%%%%%%%%%%

\begin{document}
\begin{CJK*}{UTF8}{gbsn}

  %----------HEADING----------
  % \begin{tabular*}{\textwidth}{l@{\extracolsep{\fill}}r}
  %   \textbf{\href{http://sourabhbajaj.com/}{\Large Sourabh Bajaj}} & Email : \href{mailto:sourabh@sourabhbajaj.com}{sourabh@sourabhbajaj.com}\\
  %   \href{http://sourabhbajaj.com/}{http://www.sourabhbajaj.com} & Mobile : +1-123-456-7890 \\
  % \end{tabular*}

  \begin{center}
    \textbf{\Huge \scshape 张子敬} \\ \vspace{1pt}
    \small +1 (438) 630-2638 $|$ +86 (131) 2793 0738 $|$ \href{mailto:sam.zijing.zhang@gmail.com}{\underline{sam.zijing.zhang@gmail.com}}  \\
    \href{https://linkedin.com/in/zhang-sam}{\underline{linkedin.com/in/zhang-sam}} $|$
    \href{https://github.com/SamZhang02}{\underline{github.com/SamZhang02}} $|$
    \href{https://cs.mcgill.ca/~szhang139}{\underline{cs.mcgill.ca/$\sim$szhang139}}
  \end{center}

  %-----------EDUCATION-----------
  \section{教育经历}
  \resumeSubHeadingListStart
  \resumeSubheading
  {麦吉尔大学}{加拿大, 蒙特利尔}
  {本科, 双学位:计算机科学,统计}{2022年9月 -- 2025年5月 (预计)}
  \resumeItem{GPA: 3.94/4.0}
  \resumeSubHeadingListEnd

  \section{工作经验}
  \resumeSubHeadingListStart

  \resumeSubheading
  {{全栈开发工程师实习生} $|$ \emph{TypeScript, ReactJS, C\#, ASP.NET, SQL}}
  {2023年9月至今}
  {Ceridian HCM}{加拿大, 多伦多}
  \resumeItemListStart
  \resumeItem{职位管理组}
  \resumeItem{为Ceridian的职位管理功能进行全栈开发。使用 \ \textbf{C\#} 和\ \textbf{ASP.NET} 进行后端微服务开发,和\ \textbf{TypeScript React} 进行前端开发。}
  \resumeItem{协助队伍完成了多项功能迭代,例如重塑前端UI,为后端添加添加Kafka服务等。}
  \resumeItem{为队伍清除了技术债,编写了前后端的整合测试,从而提高了团队的工作效率和代码稳定性。}
  \resumeItemListEnd
  \resumeSubheading
  {{算法实习生}$|$ \emph{Hadoop, Python, Scala, TensorFlow}}
  {2023年5月 - 2023年8月}
  {美团点评}{中国, 上海}
  \resumeItemListStart
  \resumeItem{广告算法部\  - 推荐广告质量预估组}
  \resumeItem{
    在点评APP首页广告位CTR模型上为两个特征迭代进行了离线实验。分析实验结果,提出改进方案,从而提高广告推荐系统的准确性。
  }
  \resumeItem{
    使用了\textbf{Hadoop(Hive, Spark, MapReduce, ...)生态链}进行了大数据处理,以及\textbf{TensorFlow}进行了模型调参和训练,从而产出了两个可部署的CTR模型,用于进行线上实时AB测试。
  }
  \resumeItemListEnd
  \resumeSubHeadingListEnd

  %-----------PROJECTS-----------
  \section{开源项目}
  \resumeSubHeadingListStart
  \resumeProjectHeading
  {\href{https://mcgill.courses}{\textbf{mcgill.courses}} $|$ \emph{TypeScript, React, Rust, MongoDB}}{2023年4月至今}
  \resumeItemListStart
  \resumeItem{为麦吉尔大学学生创建的开源,非官方课程搜索引擎。用于简化查询麦吉尔大学的课程和教师信息的过程。}
  \resumeItem{主要负责使用\textbf{TypeScript React}的前端构架与开发,保证用户在服务器端可以快速的得到所需的信息 。}
  \resumeItem{与使用\textbf{Rust}和\textbf{MongoDB}开发的后端使用REST API进行交互。}
  \resumeItemListEnd
  \resumeProjectHeading
  {\href{https://github.com/SamZhang02/obsidian-latex-algorithms}{\textbf{Obsidian Latex Algorithms}} $|$ \emph{TypeScript}}{2022年12月}
  \resumeItemListStart
  \resumeItem{使用\textbf{TypeScript}开发了一个「Obsidian 」Markdown 编辑器的开源插件,简化了在编辑器上编写数学证明和伪代码的过程。}
  \resumeItem{1700 + 下载数。}
  \resumeItemListEnd
  \resumeSubHeadingListEnd

  %
  %-----------PROGRAMMING SKILLS-----------
  \section{技能}
  \begin{itemize}[leftmargin=0.15in, label={}]
    \small{\item{
          \textbf{编程语言}{: Python, SQL, JavaScript, TypeScript, Java, C++, Scala} \\
          \textbf{数据库}{: SQLite, PostgreSQL, MongoDB, HiveQL}\\
          \textbf{框架}{: React, TailwindCSS, TensorFlow}\\
          \textbf{工具}{: Git, HTML/CSS, \LaTeX, Node.JS, HDFS}\\
          \textbf{语言}{: 英语, 法语, 国语, 粤语}\\
          }}
  \end{itemize}

  %-------------------------------------------
\end{CJK*}
\end{document}
